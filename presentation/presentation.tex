\documentclass{beamer}
\usetheme{PaloAlto}

\title[]{Using GitHub Professionally}

\author[]{Liam~Wrubleski\inst{1} \and Marc~Wrubleski\inst{2}}

\institute{
\inst{1}%
BSc. Electrical Engineering, BSc. Mathematics
\and
\inst{2}%
Technical Manager at the Schulich School of Engineering % TODO: actual thing here?
}

\begin{document}
\begin{frame}[plain]
  \maketitle
\end{frame}

\begin{frame}
  \frametitle{Outline}
  \tableofcontents
\end{frame}

\section{What is Github?}
\begin{frame}
  \frametitle{What is Github?}
  What is Git?\pause
  \begin{itemize}[<+->]
    \item Code repository
    \item Version control system
    \item Overall: a tool for development
  \end{itemize}

  \only{What's the Hub?}<+->
  \begin{itemize}[<+->]
    \item The largest Git server in the world
    \item Showcase of your skills and abilities
    \item History of activity
    \item Collaboration % TODO: what am I saying here?
  \end{itemize}
\end{frame}

\section{Why should I care?}
\begin{frame}
  \frametitle{Why should I care?}
  % TODO: intro Marc
  % TODO: Marc talking points
\end{frame}

\section{How do I use GitHub?} % GitHub Desktop
\begin{frame}
  \frametitle{How do I use GitHub?}
  \framesubtitle{The Basics}
  On GitHub, your work is organized into repositories, or "repos".\\\pause
  When you start a new project, you start a new repo.\\\pause
  % Here we show starting a new repo
  Many ways to work with repos:\pause
  \begin{itemize}[<+->]
    \item In-browser
    \item GitHub Desktop
    \item Command line
  \end{itemize}
\end{frame}

\begin{frame}
  \frametitle{How do I use GitHub?}
  \framesubtitle{Branches}
  GitHub lets you create "branches" on your project\\\pause
  So far, we've been working on the "master" branch\\
  \begin{alertblock}{Push to master?}
    Working on the master branch is generally strongly discouraged. Think of pushing to the master branch as releasing a new version, you don't want to do it until you're sure it's ready.
  \end{alertblock}\pause
  Branches start out identical to the master when created, and you can do as much work on each branch as you want
\end{frame}

\begin{frame}
  \frametitle{How do I use GitHub?}
  \framesubtitle{Merging}
  When you're happy with the work you've done on a branch, you can merge that branch back into "master".\\\pause
  This lets you work on multiple parts of the project at the same time, until you're sure that each part is working.\\\pause
  This can lead to an issue called a "merge conflict", which is too complicated to explain thoroughly.
\end{frame}

\begin{frame}
  \frametitle{How do I use GitHub?}
  \framesubtitle{Forking and pull requests}
  GitHub was built by and for the open source community, so anyone can contribute to pretty much anything!\\\pause
  If you want to contribute to a project, here's how you should do that:
  \begin{enumerate}[<+->]
    \item Fork
    \item Branch
    \item Work
    \item Pull request
  \end{enumerate}
\end{frame}

\section{How do I make a good profile?}
\begin{frame}
  \frametitle{How do I make a good profile?}
  \framesubtitle{Contribution History}
  Your GitHub account tracks your contribution history\\\pause
  Having a long and consistent contribution history is extremely valuable\\\pause
  The absolute best thing you can do for your profile is to consistently work on things!
\end{frame}

\begin{frame}
  \frametitle{How do I make a good profile?}
  \framesubtitle{Information}
  \textbf{What should you set?}\pause
  \begin{itemize}[<+->]
    \item Name
    \item Contact information
    \item Biography
    \begin{itemize}
      \item Short (160 characters)
      \item Grab the reader's attention
      \item Highlight your skills, education, or passions
    \end{itemize}
  \end{itemize}
\end{frame}

\begin{frame}
  \frametitle{How do I make a good profile?}
  \framesubtitle{Repositories}
  Your repositories all should have a README.md, and having nice READMEs indicates that you can keep good documentation.\\\pause
  You can have up to 6 of your public repositories pinned on your profile.\pause\\
  \textbf{What should you pin?}
  \begin{itemize}[<+->]
    \item Projects emphasizing your skills
    \item Projects you're proud of
    \item Popular repositories
  \end{itemize}
\end{frame}

\begin{frame}
  \frametitle{How do I make a good profile?}
  \framesubtitle{README.md}
  A secret feature of GitHub that gives you some more flexibility.\\
  You can use the same markdown as for your repositories, and it will show up on your profile page.
  % TODO: reduce wordiness
\end{frame}

\begin{frame}
  \frametitle{How do I make a good profile?}
  \framesubtitle{What not to include}
  Don't include more non-professional information than necessary\\\pause
  This accomplishes two things:
  \begin{itemize}[<+->]
    \item Keeps professional tools professional
    \item Minimizes the influence of any unconscious bias your reader might have
  \end{itemize}
\end{frame}

\section{Work with it!} % TODO: rename
\begin{frame}
  \frametitle{Questions?}
  % TODO: something here, probably
\end{frame}

\begin{frame}[plain]
  Thanks!
\end{frame}
\end{document}
